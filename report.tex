
%report.tex

%specify class of document
\documentclass{article}

%specify packages used 
\usepackage{graphicx}                  %use package for diagram 
\usepackage{listings}                  %use package for c++ code in appendix
\usepackage{colour}                    %use package for code listings
\usepackage[utf8]{inputenc}            %use package for colours in code
%\usepackage{url}                       %use package to display urls properly
\usepackage{amsmath}                   %use package for implies symbol
\usepackage{bm}			       %use package for bolds in math mode
\usepackage{hyperref}                  %use package for hyperlinks
%\usepackage[nottoc,numbib]{tocbibind}  %use package to include references in toc
\usepackage{algpseudocode}             %use package for pseudocode
\usepackage{algorithm}                 %use package for pseudocode
%\usepackage{array}                     %use package for centering table contents
%\usepackage{caption}		        %use package for multiple figures
%\usepackage{subcaption}                %use package for multiple figures
\usepackage{fullpage}                  %use package for full page size

%set style of bibliography
\bibliographystyle{unsrt}

%define colours for code listings
\definecolor{codegreen}{rgb}{0,0.6,0}
\definecolor{codegray}{rgb}{0.5,0.5,0.5}
\definecolor{codepurple}{rgb}{0.58,0,0.82}
\definecolor{backcolour}{rgb}{0.95,0.95,0.92}

%define a style for the listings
\lstdefinestyle{mystyle}{
    backgroundcolor=\color{backcolour},
    commentstyle=\color{codegreen},
    keywordstyle=\color{magenta},
    numberstyle=\tiny\color{codegray},
    stringstyle=\color{codepurple},
    basicstyle=\footnotesize,
    breakatwhitespace=false,
    breaklines=true,
    captionpos=b,
    keepspaces=true,
    numbers=left,
    numbersep=5pt,
    showspaces=false,
    showstringspaces=false,
    showtabs=false,
    tabsize=2
}

%set style of listing to that defined above
\lstset{style=mystyle}

%set style of footnotes
\renewcommand{\thefootnote}{\roman{footnote}}

%define command shortcuts
\newcommand{\be}{\begin{equation}}
\newcommand{\ee}{\end{equation}}

%declare hyperbolic secant
\DeclareMathOperator{\sech}{sech}

%define algorithm command
\algdef{SE}[DOWHILE]{Do}{doWhile}{\algorithmicdo}[1]{\algorithmicwhile\ #1}

%start of document
\begin{document}

%display title, my details and current date
\title{On the analytical and numerical solutions of Laplace's equation for two electrostatic configurations}
\author{S. Brown, F. Hayes, L. Heikkil{\"a}, W. Liu, D. Richardson\\
	School of Physics and Astronomy,\\
	University of Glasgow,\\
	Glasgow, United Kingdom}
\date{\today}
\maketitle

%start of abstract
\begin{abstract}

The behaviour of the electric field under the presence and influence of solid
conducting objects was studied and modelled. The Laplace equation---namely
$\nabla ^2 \phi = 0$---was solved to find the electrostatic potential $\phi$,
and hence the electric field, for two different configurations of
electrostatic systems. 

The first system had an analytical solution but both systems were solved
numerically using finite difference methods to approximate the Laplace equation.

\end{abstract} % end of abstract

%add contents
\tableofcontents

\newpage

%start of introduction
\section{Introduction}

Gauss' Law may used to express the relation between the \emph{divergence} of the
electric field, represented by \textbf{E}, and the \emph{charge density} $\rho$,
where $\epsilon_0$ is the \emph{permittivity of free space}:

\be
\nabla \cdot \bm{E} = \frac{\rho}{\epsilon_0}
\ee

Alternatively (c.f. the Divergence Theorem), it can be expressed in its
integral form, as:

\be
\oint \limits_S \bm{E} \cdot d\bm{A} = \frac{Q}{\epsilon_0}
\ee

where the integral is taken over a closed surface, $S$.

In the absence of charge, this reduces to the statement the electric field
is \emph{solenoidal}.

One can also relate the \emph{curl} of the electric field to the time
derivative of the magnetic field, denoted \textbf{B}:

\be
\nabla \times \bm{E} = - \frac{\partial \bm{B}}{\partial t}
\ee

When the \textbf{E} and \textbf{B} fields (specifically magnetic) have reached
a steady-state, in other words unchanging with time, after some initial disturbance,
this equation reduces to the statement that the electric field is \emph{irrotational}.
A consequence of irrotationality is that the electric field may then be written as negative
the gradient of some scalar potential, $\phi$ say. Mathematically:

\be
\bm{E} = -\nabla \phi
\ee

This scalar potential is called the \emph{electrostatic potential}.
One then has Laplace's equation:

\be
\nabla^2 \phi = 0
\ee

where
$\nabla^2 \equiv \frac{\partial^2}{\partial x^2}+\frac{\partial^2}{\partial y^2}+\frac{\partial^2}{\partial z^2}$ 
is the \emph{Laplacian operator}.

\newpage

%start of section
\section{Finite Difference Schemes}
Often differential equations, such as the Laplace equation, cannot be solved
analytically, and their solution must be numerically approximated.

To numerically approximate the solution to a differential equation it is
necessary to approximate derivatives. Finite difference schemes do this
by using explicit differencing to step the function from one value to the
next in small increments of some variable. The smaller the increment used,
the more accurate the solution, but the longer and more computationally
intensive the process.

Suppose one wishes to approximate the first derivative of a function $f(t)$, say,
with respect to the variable $t$. The simplest way to do this is to discretise the
derivative and write

\be
\frac{df}{dt} \approx \frac{f(t+\Delta t) - f(t)}{\Delta t}
\ee

where $f(t+\Delta t)$ and $f(t)$ are two values of $f$, evaluated at two points
a distance $\Delta t$ apart.

To approximate the second derivative of a function, one writes

\be
\frac{d^2 f}{dt^2} \approx \frac{f(t)' - f(t-h)'}{\Delta t} \\
= \frac{\frac{f(t +\Delta t) - f(t)}{\Delta t} - \frac{f(t) - f(t -\Delta t)}{\Delta t}}{\Delta t} \\
= \frac{f(t +\Delta t) - 2f(t) + f(t-\Delta t)}{\Delta t ^2}
\ee

We wish to discretise the following equation:

\be
\frac{\partial^2 \phi}{\partial x^2}+\frac{\partial^2 \phi}{\partial y^2} = 0
\ee

One can approximate this partial differential equation as:

\be
\frac{u_{j+1,k}-2u_{j,k}+u_{j-1,k}}{\Delta x^2} = - \frac{u_{j,k+1}-2u_{j,k}+u_{j,k-1}}{\Delta y^2}
\ee

where $j$ and $k$ index the $x$ and $y$ directions respectively.

If we use the same step-size $\Delta=\Delta x=\Delta y$ in both $x$ and $y$ directions this
reduces to:

\be
u_{j,k}= \frac{1}{4}(u_{j-1,k}+u_{j+1,k}+u_{j,k-1}+u_{j,k+1})
\ee

In other words, the value of the potential at each point is the average of
the four surrounding points.

\newpage

%start of section
\section{System A}

Suppose one has a perfectly uniform electric field, between two infinite planes
at potentials $V$ and $-V$, a distance $2d$ apart. Suppose one then places an
infinitely long perfectly conducting cylinder into the centre of the field,
at ground potential. We wish to find the resulting form of the electrostatic
potential, and hence the electric field, surrounding the cylinder.

\subsection{Analytic Solution}
First, one realises that the three-dimensional problem can be reduced to an entirely
equivalent one in two dimensions due to the translation symmetry of the system along
the length of the cylinder. So, we consider a cross-section of the system and introduce
a polar co-ordinate system, with origin centred on the centre of the cylinder.

Laplace's equation in polar co-ordinates is: 

\be
\frac{\partial^2 \phi}{\partial r^2}+\frac{1}{r}\frac{\partial \phi}{\partial r}+\frac{1}{r^2}\frac{\partial^2 \phi}{\partial \theta^2}
= \frac{1}{r}\frac{\partial}{\partial r}(r \frac{\partial \phi}{\partial r}) + \frac{1}{r^2}\frac{\partial ^2 \phi}{\partial \theta^2}
= 0
\ee

We employ separation of variables and posit a solution of the form
$\phi = f(r)g(\theta)$ for two unknown functions $f$ and $g$. Upon substitution,
one finds that:

\be
\frac{r}{f(r)}\frac{d}{dr}(r \frac{df(r)}{dr}) =- \frac{1}{g(\theta)}\frac{d^2 g(\theta)}{d\theta^2}
\ee

Since this is true for arbitrary values of $r$ and $\theta$, we set:

\be
\frac{r}{f(r)}\frac{d}{dr}(r \frac{df(r)}{dr}) =- \frac{1}{g(\theta)}\frac{d^2 g(\theta)}{d\theta^2}=k^2
\ee

for some constant $k$.

This gives two second order partial differential equations:

\be
r\frac{d}{dr}(r \frac{df(r)}{dr}) = k^2 f(r) \qquad
\frac{d^2 g(\theta)}{d\theta^2}=-k^2 g(\theta)
\ee

For the case $k=0$, these equations have solutions
$f(r)=\alpha \ln(r) + \beta$ and $g(\theta) = \gamma \theta + \delta$.
For non-zero $k$, they have solutions

\be
f(r)=\alpha_k r^k + \beta_k r^{-k}
\qquad
g(\theta)= \gamma_k \sin(k\theta)+\delta_k \cos(k\theta)
\ee

With the sanity requirement that $g(\theta)=g(\theta + 2\pi)$, we have that $k$ is
an integer. Hence, by the principle of superposition and the linearity of differentiation,
the general solution to the Laplace equation in polar co-ordinates is a sum of
these terms:

\be
\phi(r,\theta)
= f(r)g(\theta)
= (\alpha \ln(r) + \beta)(\gamma\theta + \delta) + \sum_{n=1}^{\infty}(\alpha_n r^n+\beta_n r^{-n})(\gamma_n \sin(n\theta) + \delta_n \cos(n\theta))
\ee

For a particular solution to the system considered here one must impose boundary
conditions.

By considering the geometry of the system, we expect a solution that is symmetric
about $\theta=0$. This implies that $\gamma = 0$ and $\gamma_n=0$ $\forall n$ as
$\sin()$ is anti-symmetric about the origin (c.f. odd).

Additionally, the potential is finite as $r \rightarrow \infty$, implying that
$\alpha$ and $\alpha_n$ are both zero. We now have:

\be
\phi(r,\theta)=\beta + \sum_{n=1}^{\infty}(\frac{\beta_n}{r^n} \cos(n\theta))
\ee

where the $\beta$'s have absorbed other constants.

In particular, as $r \rightarrow \infty$, we require
$\phi=-\frac{V}{d}x=-\frac{V}{d}r\cos(\theta)$. Since the infinite sum vanishes,
$\beta=-\frac{V}{d}r\cos(\theta)$.

We now have

\be
\phi(r,\theta)=-\frac{V}{d}r\cos(\theta) + \sum_{n=1}^{\infty}(\frac{\beta_n}{r^n} \cos(n\theta))
              =(\frac{\beta_1}{r}-\frac{V}{d}r)\cos(\theta) + \sum_{n=2}^{\infty}(\frac{\beta_n}{r^n} \cos(n\theta))
\ee

We require $\phi=0$ at $r=a$ so that $\beta_1=\frac{Va^2}{d}$ and $\beta_{n \geq 2}=0$

Thus, the final form for the electrostatic potential is, for a cylinder of radius $a$:

\be
\phi(x,y)=
\begin{cases} \qquad 0, & \quad r \leq a \\
\frac{V}{d}(\frac{a^2}{r}-r)\cos(\theta), & \quad r > a
\end{cases}
\ee

\subsection{Numerical Solution}

The algorithm employed to find the potential for System A was

\begin{algorithm}
\begin{algorithmic}[]
\Procedure{Finite Difference Method}{}
\State declare variables:
\State $V \gets$ potential on plates
\State $\delta \gets$ position step-size
\State $d \gets$ distance between plates
\State $h \gets$ height of plates
\State $r \gets$ radius of cylinder
\State specify initial system:
\For {$j=1$ to $\frac{d}{\delta}$}
   \For {$k=1$ to $\frac{d}{\delta}$}
      \State $u_{j, k} \gets V-\frac{2Vj\delta}{d}$
   \EndFor
\EndFor
\State find solution:
\For {$j=2$ to $\frac{d}{\delta}-1$}
   \For {$k=2$ to $\frac{h}{\delta}-1$}
      \If {$(j \delta-\frac{1}{2} d)^2+(k \delta-\frac{1}{2} h)^2<r^2$}
         \State $u_{j, k} \gets 0$
      \Else
         \State $u_{j,k} \gets \frac{1}{4}(u_{j-1,k}+u_{j+1,k}+u_{j,k-1}+u_{j,k+1})$
      \EndIf
   \EndFor
\EndFor
\State plot solution
\State find electric field:
\For {$j=1$ to $\frac{d}{\delta}-1$}
   \For {$k=1$ to $\frac{h}{\delta}-1$}
      \State $x\_grad_{j, k} \gets \frac{u_{j+1,k}-u_{j,k}}{dx}$
      \State $y\_grad_{j,k} \gets \frac{u_{j,k+1}-u_{j,k}}{dy}$
   \EndFor
\EndFor
\EndProcedure
\end{algorithmic}
\end{algorithm}

\newpage

%start of section
\section{System B}

\subsection{Numerical Solution}

\newpage

%start of section
\section{Conclusion}

\newpage

%include BibTex bibliography
\bibliography{buffon}

\newpage

%start of appendix
\appendix
\section{Appendices}
\subsection{Code} \label{app:code}
This is the algorithm implemented in C++:
%\lstinputlisting[language=C++]{/home/danr/}

\end{document} %end of document
