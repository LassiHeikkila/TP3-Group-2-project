%report.tex

%specify class of document
\documentclass[12pt, a4paper]{article}

%specify packages used 
\usepackage{graphicx}                  %use package for diagram 
\usepackage{listings}                  %use package for c++ code in appendix
\usepackage{colour}                    %use package for code listings
\usepackage[utf8]{inputenc}            %use package for colours in code
%\usepackage{url}                       %use package to display urls properly
\usepackage{amsmath}                   %use package for implies symbol
\usepackage{bm}			       %use package for bolds in math mode
\usepackage{hyperref}                  %use package for hyperlinks
\usepackage{algpseudocode}             %use package for pseudocode
\usepackage{algorithm}                 %use package for pseudocode
%\usepackage{array}                     %use package for centering table contents
%\usepackage{caption}		        %use package for multiple figures
%\usepackage{subcaption}                %use package for multiple figures
\usepackage{fullpage}                  %use package for full page size
\usepackage{tikz}		       %use package for diagram drawings
\usepackage{graphicx}		       %use package for figures

%set style of bibliography
\bibliographystyle{unsrt}

%define colours for code listings
\definecolor{codegreen}{rgb}{0,0.6,0}
\definecolor{codegray}{rgb}{0.5,0.5,0.5}
\definecolor{codepurple}{rgb}{0.58,0,0.82}
\definecolor{backcolour}{rgb}{0.95,0.95,0.92}

%define a style for the listings
\lstdefinestyle{mystyle}{
    backgroundcolor=\color{backcolour},
    commentstyle=\color{codegreen},
    keywordstyle=\color{magenta},
    numberstyle=\tiny\color{codegray},
    stringstyle=\color{codepurple},
    basicstyle=\footnotesize,
    breakatwhitespace=false,
    breaklines=true,
    captionpos=b,
    keepspaces=true,
    numbers=left,
    numbersep=5pt,
    showspaces=false,
    showstringspaces=false,
    showtabs=false,
    tabsize=2
}

%set style of listing to that defined above
\lstset{style=mystyle}

%set style of footnotes
\renewcommand{\thefootnote}{\roman{footnote}}

%define command shortcuts
\newcommand{\be}{\begin{equation}}
\newcommand{\ee}{\end{equation}}

%start of document
\begin{document}

%display title, my details and current date
\title{On the analytical and numerical solutions of Laplace's equation for different electrostatic configurations}
\author{S. Brown, F. Hayes, L. Heikkil{\"a}, W. Liu, D. Richardson\\
	School of Physics and Astronomy,\\
	University of Glasgow,\\
	Glasgow, United Kingdom}
\date{\today}
\maketitle

%start of abstract
\begin{abstract}

The behaviour of the electric field under the presence and influence of solid
conducting objects is modelled and studied. Finite difference methods, namely the
relaxation method, are employed to solve the Laplace equation numerically,
to find the approximate form of the electrostatic potential, and hence the electric
field, present in various electrostatic systems. We explicitly focus on two such
systems: the case of a long perfectly conducting cylinder centred between two infinite
planes at potential $+V$ and $-V$; and the case of a silicon detector---a system
composed of two silicon wafers with segmented doped implants at ground potential on one
side and a uniform doped implant, referred to as the backplane, on the other side, held
at a potential $+V$. We compare the analytical solution of the first system to its
approximate numerical solution, and study the convergence of the numerical method.

\end{abstract} % end of abstract

%start of introduction
\section{Introduction}
\subsection{Derivation of Laplace's Equation}

Gauss' Law may used to express the relation between the \emph{divergence} of the
electric field, represented by \textbf{E}, and the \emph{charge density} $\rho$,
where $\epsilon_0$ is the \emph{permittivity of free space}:
%
\be
\nabla \cdot \bm{E} = \frac{\rho}{\epsilon_0}
\ee
%
Alternatively (c.f. the Divergence Theorem), it can be expressed in its
integral form, as:
%
\be
\oint \limits_S \bm{E} \cdot d\bm{A} = \frac{Q}{\epsilon_0}
\ee
%
where the integral is taken over a closed surface, $S$, and $Q$ is the enclosed charge.

In the absence of charge, as is the case outside a conductor, this reduces to the
statement the electric field is \emph{solenoidal}.

One can also relate the \emph{curl} of the electric field to the time
derivative of the magnetic field, denoted \textbf{B}:
%
\be
\nabla \times \bm{E} = - \frac{\partial \bm{B}}{\partial t}
\ee

If the magentic field is constant, in other words unchanging with time, this equation
reduces to the statement that the electric field is \emph{irrotational}. A
consequence of irrotationality is that the electric field may then be written as
negative the gradient of some scalar potential, $\phi$ say. Mathematically:
%
\be
\bm{E} = -\nabla \phi
\ee
%
This scalar potential is called the \emph{electrostatic potential}.
One then has \emph{Laplace's equation}:
%
\be
\nabla^2 \phi = 0
\ee
%
where
$\nabla^2 \equiv \frac{\partial^2}{\partial x^2}+\frac{\partial^2}{\partial y^2}+\frac{\partial^2}{\partial z^2}$ 
is the \emph{Laplacian operator}\footnote{Specifically in Cartesian co-ordinates}.

This is a second-order partial differential equation that can be solved, given some
well-posed boundary conditions, to find the electrostatic potential and electric field
for some physical system.

\subsection{Finite Difference Schemes}

It is often the case that differential equations, such as the Laplace equation, cannot
be solved analytically, and their solution must be numerically approximated.

To numerically approximate the solution to a differential equation it is
necessary to approximate the derivative of a function. The usual method employed to
do this is finite differencing. This approximates the derivative by using explicit
differencing to step the function from one value to the next in small 
increments of some variable. The smaller the increment used, the more accurate
the approximation, but the longer and more computationally intensive the process.

Suppose one wishes to approximate the first derivative of a function $f(t)$, say,
with respect to the variable $t$. The simplest way to do this is to discretise the
derivative and write
%
\be
\frac{df}{dt} \approx \frac{f(t+\Delta t) - f(t)}{\Delta t}
\ee
%
where $f(t+\Delta t)$ and $f(t)$ are two values of $f$, evaluated at two points
a distance $\Delta t$ apart.

To approximate the second derivative of a function, one writes
%
\be
\frac{d^2 f}{dt^2} \approx \frac{f'(t) - f'(t-h)}{\Delta t} \\
= \frac{\frac{f(t +\Delta t) - f(t)}{\Delta t} - \frac{f(t) - f(t -\Delta t)}{\Delta t}}{\Delta t} \\
= \frac{f(t +\Delta t) - 2f(t) + f(t-\Delta t)}{\Delta t ^2}
\ee

\subsubsection{Derivation of the Relaxation Method}

For our purposes, we wish to discretise the following equation:
%
\be
\frac{\partial^2 \phi}{\partial x^2}+\frac{\partial^2 \phi}{\partial y^2} = 0
\ee
%
One can approximate this partial differential equation as:
%
\be
\frac{\phi_{j+1,k}-2\phi_{j,k}+\phi_{j-1,k}}{\Delta x^2} = - \frac{\phi_{j,k+1}-2\phi_{j,k}+\phi_{j,k-1}}{\Delta y^2}
\ee
%
where $j$ and $k$ index the $x$ and $y$ directions respectively.

If we use the same step-size $\Delta=\Delta x=\Delta y$ in both $x$ and $y$ directions this
reduces to:
%
\be
\phi_{j,k}= \frac{1}{4}(\phi_{j-1,k}+\phi_{j+1,k}+\phi_{j,k-1}+\phi_{j,k+1})
\ee
%
In other words, the value of the potential at a specific point is the average of
the four surrounding points. This is the relaxation method.

%start of section
\section{System A}

Suppose one has a perfectly uniform electric field between two infinite planes
at potentials $V$ and $-V$, and they are a distance $2d$ apart. Suppose one then
places an infinitely long perfectly conducting cylinder, of radius $R$, into the centre
of the field, at ground potential. We wish to find the resulting form of the
electrostatic potential, and hence the electric field, surrounding the cylinder.

\begin{figure}[h!]
\begin{center}
\begin{tikzpicture}
\draw (0,0) node[below] {$x = -d$} -- (0,6) node[above] {$\phi = V$}; 
\draw (10,0) node[below] {$x = d$} -- (10,6) node[above] {$\phi = -V$}; 
\draw (5,3) circle (1cm) node[below] {$\phi=0$};
\draw[dashed] (5,3) -- (5.6,3.8) node[pos=0.5,above left] {$R$};
\end{tikzpicture}
\end{center}
\caption{Cross-sectional diagram of System A}
\end{figure}

\subsection{Analytic Solution}

First, one realises that the three-dimensional problem can be reduced entirely to two
dimensions due to the translation symmetry of the system along the length of the
cylinder. So, we consider a cross-section of the system and introduce a polar
co-ordinate system, with origin centred on the centre of the cylinder.

Laplace's equation in polar co-ordinates is: 
%
\be
\frac{\partial^2 \phi}{\partial r^2}+\frac{1}{r}\frac{\partial \phi}{\partial r}+\frac{1}{r^2}\frac{\partial^2 \phi}{\partial \theta^2}
= \frac{1}{r}\frac{\partial}{\partial r}(r \frac{\partial \phi}{\partial r}) + \frac{1}{r^2}\frac{\partial ^2 \phi}{\partial \theta^2}
= 0
\ee

We employ separation of variables and posit a solution of the form
$\phi = f(r)g(\theta)$ for two unknown functions $f$ and $g$. Upon substitution,
one finds that:
%
\be
\frac{r}{f(r)}\frac{d}{dr}(r \frac{df(r)}{dr}) =- \frac{1}{g(\theta)}\frac{d^2 g(\theta)}{d\theta^2}
\ee

Since this is true for arbitrary values of $r$ and $\theta$, we set:
%
\be
\frac{r}{f(r)}\frac{d}{dr}(r \frac{df(r)}{dr}) =- \frac{1}{g(\theta)}\frac{d^2 g(\theta)}{d\theta^2}=k^2
\ee
%
for some constant $k$.

This gives two second order partial differential equations:
%
\be
r\frac{d}{dr}(r \frac{df(r)}{dr}) = k^2 f(r) \qquad
\frac{d^2 g(\theta)}{d\theta^2}=-k^2 g(\theta)
\ee

For the case $k=0$, these equations have solutions
$f(r)=\alpha \ln(r) + \beta$ and $g(\theta) = \gamma \theta + \delta$.
For non-zero $k$, they have solutions
%
\be
f(r)=\alpha_k r^k + \beta_k r^{-k}
\qquad
g(\theta)= \gamma_k \sin(k\theta)+\delta_k \cos(k\theta)
\ee

With the physically reasonable requirement that $g(\theta)=g(\theta + 2\pi)$, we have
that $k$ is an integer. Hence, by the principle of superposition and the linearity of
differentiation, the general solution to the Laplace equation in polar co-ordinates is
a sum of these terms:
%
\be
\phi(r,\theta)
= f(r)g(\theta)
= (\alpha \ln(r) + \beta)(\gamma\theta + \delta) + \sum_{n=1}^{\infty}(\alpha_n r^n+\beta_n r^{-n})(\gamma_n \sin(n\theta) + \delta_n \cos(n\theta))
\ee

For a particular solution to the system considered here one must impose boundary
conditions.

By considering the geometry of the system, we expect a solution that is symmetric
about $\theta=0$. This implies that $\gamma = 0$ and $\gamma_n=0$, $\forall n$ as
$\sin()$ is anti-symmetric about the origin (c.f. odd). Additionally, the potential
is finite as $r \rightarrow \infty$, implying that $\alpha$ and $\alpha_n$ are both
zero. We now have:
%
\be
\phi(r,\theta)=\beta + \sum_{n=1}^{\infty}(\frac{\beta_n}{r^n} \cos(n\theta))
\ee
%
where the $\beta$'s have absorbed other constants. In particular, as
$r \rightarrow \infty$, we require $\phi=-\frac{V}{d}x=-\frac{V}{d}r\cos(\theta)$.
Since the infinite sum vanishes, at infinity, $\beta=-\frac{V}{d}r\cos(\theta)$.

We now have that
%
\be
\phi(r,\theta)=-\frac{V}{d}r\cos(\theta) + \sum_{n=1}^{\infty}(\frac{\beta_n}{r^n} \cos(n\theta))
              =(\frac{\beta_1}{r}-\frac{V}{d}r)\cos(\theta) + \sum_{n=2}^{\infty}(\frac{\beta_n}{r^n} \cos(n\theta))
\ee

We require that the potential is zero on the surface of the cylinder,
$\phi(R,\theta)=0$, so that $\beta_1=\frac{VR^2}{d}$ and $\beta_{n \geq 2}=0$

Thus, the final form for the electrostatic potential is
%
\be
\phi(x,y)=
\begin{cases} \qquad 0, & \quad r \leq a \\
\frac{V}{d}(\frac{R^2}{r}-r)\cos(\theta), & \quad r > a
\end{cases}
\ee

\subsection{Numerical Solution}

To numerically approximate the electrostatic potential for System A via the 
relaxation method the following rudimentary algorithm was developed.

\begin{algorithm}
\begin{algorithmic}[]
\Procedure{The Relaxation Method}{}
\State declare variables:
\State $V \gets$ potential on plates
\State $\delta \gets$ position step-size
\State $d \gets$ distance between plates
\State $h \gets$ height of plates
\State $r \gets$ radius of cylinder
\State $its \gets$ number of iterations
\State $nx \gets \frac{d}{\delta}$ 
\State $ny \gets \frac{h}{\delta}$ 
\State specify boundary potentials:
\For {$j=1$ to $ny$}
   \State $u_{j, 1} = +V$
   \State $u_{j, nx} = -V$
\EndFor
\For {$k=1$ to $nx$}
   \State $u_{1, k} \gets V-\frac{2Vj}{nx}$
   \State $u_{ny, k} \gets V-\frac{2Vj}{nx}$
\EndFor
\State find solution:
\For {$i=1$ to $its$}
   \For {$j=2$ to $ny-1$}
      \For {$k=2$ to $nx-1$}
         \If {$(j \delta-\frac{1}{2} d)^2+(k \delta-\frac{1}{2} h)^2<r^2$}
            \State $u_{j, k} \gets 0$
         \Else
            \State $u_{j,k} \gets \frac{1}{4}(u_{j-1,k}+u_{j+1,k}+u_{j,k-1}+u_{j,k+1})$
         \EndIf
      \EndFor
   \EndFor
\EndFor
\State find electric field:
\For {$j=1$ to $ny-1$}
   \For {$k=1$ to $nx-1$}
      \State $(Ex)_{j, k} \gets -\left(u_{j,k+1}-u_{j,k}\right)/\delta$
      \State $(Ey)_{j,k} \gets -\left(u_{j+1,1}-u_{j,k}\right)/\delta$
   \EndFor
\EndFor
\State plot potential and field
\EndProcedure
\end{algorithmic}
\end{algorithm}

\subsection{Comparison of Analytical and Numerical Solution}

%start of section
\section{System B}

The second system considered was a silicon detector---a system consisting of two
silicon wafers, one segmented with doped implants at ground potential and the other,
referred to as the backplane, uniformly doped, held at a potential $+V$.

\begin{figure}[h!]
\begin{center}
\begin{tikzpicture}
\draw (0,0) -- (12,0) node[right] {$\phi = V$}; 
\draw (0,5) -- (12,5) node[pos=0.25, above] {GND} node[pos=0.5, above] {GND} node[pos=0.75,above] {GND}; 
\draw (2.5,5) rectangle (3.5,4.75);
\draw (5.5,5) rectangle (6.5,4.75);
\draw (8.5,5) rectangle (9.5,4.75);
\end{tikzpicture}
\end{center}
\caption{Cross-sectional diagram of System B}
\end{figure}

\subsection{Numerical Solution}

%start of section
\section{Other Electrostatic Systems}

We developed a general software package in C++ to solve and plot the potential and
electric field for an arbitrary electrostatic system. It accepts input of a coloured
bitmap, where the colours represent different potentials, and returns plots of the
resultant electrostatic potential and electric field.

%start of section
\section{Conclusion}

\subsection{Further Work}

%include BibTex bibliography
\bibliography{report}

%start of appendix
\appendix
\section{Appendices}

\end{document} %end of document
