\documentclass{beamer}
\usepackage{graphicx}
\usepackage{amsmath}
\usepackage{float}
\usepackage{tikz}
\usetheme{default}
\title{Using the Laplace Equation to Solve Various Boundary Conditions}
\subtitle{Graphical Boundary Conditions and Relative Error}
\author{Fergus Hayes}
\date{\today}
\begin{document}
\begin{frame}
\titlepage
\end{frame}
\begin{frame}
\frametitle{Outline}
\tableofcontents
\end{frame}
\section{Introduction}
\begin{frame}{Introduction}
\begin{itemize}
\item The program uses the checker board with successive over relaxation (SOR) method.
\item An easy way to input boundary conditions is through image files.
\item The method requires some measure of precision. 
\end{itemize}
\end{frame}
\section{Bitmaps and Boundaries}
\subsection{Graphical Boundary Conditions}
\begin{frame}
\frametitle{Graphical Boundary Conditions}
\begin{itemize}
\item The boundary conditions for all systems can be constructed graphically on an image file. 
\item Different can colours represent different areas of constant potential.
\end{itemize}
The image file must:
\begin{itemize}
\item Allow at least four colours to be easily distinguished.
\item Be easy to extract information from.
\end{itemize}
\end{frame}
\begin{frame}
\subsection{Mapping Boundaries}
\frametitle{Mapping Boundaries}
The locations of the boundary conditions and their potential need to be mapped from the image file to two arrays.
\begin{itemize}
\item The grid of potentials to be iterated over using SOR.
\item The boolean array which stores the location of constant boundary values.
\end{itemize}
\end{frame}
\begin{frame}
\subsection{Bitmaps}
\frametitle{Bitmaps}
\begin{itemize}
\item Bitmaps (BMP files) are uncompressed image files.
\item Each pixel has 24 bits of memory for storing colour (16,777,216 different colours).
\item 8 bits of memory for each colour: red, green and blue.
\end{itemize}
\begin{figure}[H]
	\centering
	\includegraphics[width=0.5\textwidth]{colourscale.jpg}
	\caption{0 to 255 colour scale for various colours.}
\end{figure}
\end{frame}
\subsection{Bitmap Boundary Condition}
\begin{frame}
\frametitle{Bitmap Boundary Conditions}
\begin{figure}[H]
	\centering
	\includegraphics[width=0.4\textwidth]{SystemA.jpg}
	\caption{System A graphically represented as a BMP file.}
\end{figure}
\begin{itemize}
\item The white pixels are of unsolved potential while different colours are areas of different fixed potential.
\end{itemize}
\end{frame}

\begin{frame}
\frametitle{bitmaps.cpp}
'bitmaps.cpp' was used to extract BMP boundary information.
\begin{itemize}
\item Commands are used from 'bitmap\textunderscore image.hpp'.
\item The image is opened using:\\
\begin{align*}
bitmap\textunderscore image \; image('file \;name')
\end{align*}
\item The dimensions are found:\\
\begin{align*}
int\; height = image.height(),\; width = image.width()
\end{align*}
\end{itemize}
\end{frame}
\begin{frame}
\subsection{bitmaps.cpp}
\frametitle{bitmaps.cpp}
\begin{itemize}
\item Each pixels colour is analysed by looping through pixel array and using:\\
\begin{align*}
image.get\textunderscore pixel(p,q,red,green,blue)
\end{align*}
\item The colour of the pixel is decided by what variable has the largest colour scale.
\item Corresponding potential is mapped to a similar array with same coordinates.
\item Coloured pixels are set to true in the boolean array.
\end{itemize}
\end{frame}
\section{Relative Error and Convergence}
\subsection{Convergence}
\begin{frame}
\frametitle{Convergence}
\begin{itemize}
\item SOR method does not have an easily quantifiable error. So analyse how the results converge instead.
\item a numerically determined value $\tilde{\phi}$ converges to some true value $\phi$ over iterations $n$:
\begin{align*}
\lim_{n \to \infty} \tilde{\phi} = \phi
\end{align*}
\item As $\phi$ is a constant:
\begin{align*}
\lim_{n \to \infty} \frac{d \tilde{\phi}}{dn} = \lim_{n \to \infty} (\phi^{n} - \phi^{n-1}) = 0
\end{align*}
\item The magnitude of $\phi^{n} - \phi^{n-1}$, is related to the accuracy.
\end{itemize}
\end{frame}
\subsection{Relative Error}
\begin{frame}
\frametitle{Relative Error}
\begin{itemize}
\item  The relative error $\epsilon_{rel}$:
\begin{align*}
\epsilon_{rel \; i,j}^{n} = |\phi_{i,j}^{n} - \phi_{i,j}^{n-1}|
\end{align*}  
\item The lower this value, the closer the iteration is from converging to a point.
\end{itemize}
\end{frame}
\subsection{Convergence Limit}
\begin{frame}{Convergence Limit}
\begin{itemize}
\item A desired relative error, $\epsilon$, is taken in from the command line arguments.
\item The program loops until $\epsilon_{rel} < \epsilon$ for all points.
\item Within each iteration, a count increments by one for every point where the condition is met.
\item On the iteration when the count equals the number of pixels, the loop breaks. The system is then said to have converged.
\end{itemize}
\end{frame}
\section{Evaluation}
\begin{frame}
\frametitle{Evaluation}
estatics is able to..
\begin{itemize}
\item Determine the potential for many boundary conditions.
\item Find potentials to any given degree of accuracy.
\end{itemize}
Possible improvements..
\begin{itemize}

\item Allow the bitmaps to accept more boundary conditions.
\item Able to skip over points when converged.
\end{itemize}
Thanks for listening!
\end{frame}
\end{document}
