	\subsubsection{Data Structures}
	
		In order to simplify the passing of pertinent data to and from sub-functions, two structs were created. Structs - or records - are a type of data structure which can hold an arbitrary number of fields, each of which can be of a different data type. This makes them ideal for this case; arrays of type \lstinline|double| are required to hold the system to be solved and the gradient components of the electric field, \lstinline|int| variables can be used to hold the dimensions of the array - which allows the solver to simply read this information rather than calculate it when needed - and to pass back the number of iterations required to meet the desired convergence, and the \lstinline|bool| data type can be used to write-protect boundary elements. This final point is discussed in detail in section~\ref{sec:mask}.
		
		\subsubsection{Implementation}
		
		In the case of \lstinline|eStatics|, a global pointer to the data struct was defined as \lstinline|extern| in the header file and created outside the main function. This allows the designation of a memory address to the structure which can be seen by all sub-routines, allowing each to read from and write to the fields contained within it. The \lstinline|new| command is used to dynamically assign memory to those fields, and expand them when needed. This allows systems of arbitrary size, or fineness of grid, to be solved by the software package. Another benefit of using a global struct is that variables such as system dimension, desired convergence, and maximum iterations need only be written once when building the system - from then on only memory reads are required. Finally, having only one set of data to be passed by address lowers RAM overhead due to there being no need to copy data and pass it by value to a sub-routine.
