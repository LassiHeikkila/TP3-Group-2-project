The previous expression for the Laplace equation to find the potential of point $(x,y)$
using points $(x,y+h)$, $(x,y-h)$, $(x-h,y)$ and $(x+h,y)$ can be expanded to
include the points $(x+h,y+h)$, $(x+h,y-h)$, $(x-h,y+h)$ and $(x-h,y-h)$. This is
done by incorporating the Taylor expansion for the new points with the
previous expression. The Laplace equation becomes:
%
\begin{align}
\nabla ^2 \phi &= \frac{4(\phi_{i+1,j} + \phi_{i-1,j} + \phi_{i,j-1} + \phi_{i,j+1})}{6h^2} \nonumber \\
&+ \frac{\phi_{i+1,j-1} + \phi_{i+1,j+1} + \phi_{i-1,j-1} + \phi_{i-1,j+1}}{6h^2} \nonumber \\
&- \frac{20\phi_{i,j}}{6h^2} - \frac{h^2(\phi_{xxxx} + 2 \phi_{xxyy} + \phi_{yyyy})}{6} - O(h^4)
\end{align}

The truncation error associated with the expression is second order. However, the terms
inside the bracket $(\phi_{xxxx} + 2 \phi_{xxyy} + \phi_{yyyy})$ is equal to the
Laplacian of the Laplacian of the potential $\nabla^2 (\nabla ^2 \phi)$. For the case
for the Laplace equation, $\nabla ^2 \phi = 0$ and therefore the second order term in
the expression is reduced to zero and the truncation error becomes fourth order. This
allows the nine-point stencil to be more accurate with less iterations than the
five-point stencil. However, the larger calculations have a computational cost and
increases the time spent on each iteration.
