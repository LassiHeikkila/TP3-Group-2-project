Like the Gauss-Seidel method, the successive over-relaxation method uses two
points from the previous iteration and the two updated points. It also
utilises the point $\phi_{i,j,n-1}$, and applies over-relaxation by using a
constant $s$ known as the relaxation parameter.
%
\be
\phi_{j,k,n+1}=(1-s)\phi_{j,k,n}+\frac{s}{4}(\phi_{j-1,k,n+1}+\phi_{j+1,k,n}+\phi_{j,k-1,n+1}+\phi_{j,k+1,n})
\ee

The optimum value for the relaxation parameter cannot be easily determined
arithmetically but to allow the system to converge, the value is always
within $s\in[1,2]$. The optimal value for a square grid (see~\cite{sor})
can be approximated using:
%
\be
 s_{opt} = \frac{2}{1+\sin(\frac{\pi}{n})} \approx \frac{2}{1+\frac{\pi}{n}}
\ee
%
for an $n \times n$ square grid.

The method reduces iterations by a considerable amount depending on the value
of the relaxation parameter used.
