\subsection{Successive Over-Relaxation}
Solving the Laplace equation using the finite difference method can be written in the form of a one dimensional linear equation by replacing:
\be
i = j(K+1) + k
\ee
The solution of all the points on the grid is a system of linear equations which can be represented in matrix form as:
\be
\label{matrixform}
\boldmath A \cdot \phi = b
\ee
Where the values of the potential are $\boldsymbol{\phi}$ and $\boldsymbol{A}$ is a "tridiagonal with fringes" matrix. This matrix can be split into matrices $\boldsymbol{L + D + U}$, where $\boldsymbol{L}$ is the lower triangle, $\boldsymbol{D}$ is the diagonal elements  and $\boldsymbol{U}$ is the upper triangle of matrix $\boldsymbol{A}$. Equation \ref{matrixform} then becomes:
\be
\boldsymbol{(L + D + U) \cdot \phi = b}
\ee
This is then multiplied by a constant $s$ - the relaxation parameter - where $s > 1$. By also knowing  the potential on the left hand side, $\boldsymbol{\phi^{n}}$ is numerically solved using the potential determined from the previous iteration, $\boldsymbol{\phi^{n-1}}$ the expression becomes:
\be
\boldsymbol{\phi^{n} = (D + sL)^{-1}(b - (sU + (s-1)D) \cdot \phi^{n-1})}
\ee
This is in the form of $\boldsymbol{\phi^{n} = B\phi^{n-1} + b}$ where $\boldsymbol{B}$ is the matrix $\boldsymbol{(D + sL)^{-1}(sU + (s-1)D)}$. The solutions to $\phi^{n}$ converge to their true values faster with lower $\boldsymbol{B}$. Therefore the relaxation parameter is chosen to minimise $\boldsymbol{||(D + sL)^{-1}(sU + (s-1)D)}||$. In two dimensional iterative form the successive over-relaxation method for the Laplace equation becomes:
\be
\phi_{j,k,n+1}= (1-s)\phi_{j,k,n}+\frac{s}{4}(\phi_{j-1,k,n+1}+\phi_{j+1,k,n}+\phi_{j,k-1,n+1}+\phi_{j,k+1,n})
\ee
Neither the Jacobi iterative method nor the Gauss-Seidel method utilise the value
of the potential at the current point in the previous iteration. The optimum value for the relaxation parameter can not be easily determined arithmetically but the value is always within $1<s<2$ for a converging system. The optimal value for a square grid can be approximated using:
 \be
 s_{opt} = \frac{2}{1+\sin(\frac{\pi}{n})} \approx \frac{2}{1+\frac{\pi}{n}}
 \ee
For an $n x n$ square grid.
