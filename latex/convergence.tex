The successive over-relaxation method does not have an easily quantifiable error. The truncation error from the finite difference approximation is found by calculating higher derivatives than the solution requires and therefore requires substantially more computation to determine. Instead the precision of our results are analysed.

The precision is related to how close a numerically determined value is from the true value it is converging to. If a numerical process approximates a value of $x$ by returning a value $\tilde{x}$ on the $n$th iteration, then:
\be
\lim_{n \rightarrow \infty}\tilde{x} = x                     
\ee

The derivative of this expression tends towards zero as the value $x$ is constant:
\be
\lim_{n \rightarrow \infty}\frac{d\tilde{x}}{dn} = \lim_{n \rightarrow \infty} (\tilde{x}^{n} - \tilde{x}^{n-1}) = 0                     
\ee
The magnitude of the difference of determined values $\tilde{x}$ between successive iterations is then related to how close the value is to the true value $x$. The closer the value is to zero, the more precise the value.

The magnitude of the difference of the potential between successive iterations at some point is referred to as the absolute convergence $\epsilon_{abs}$ at that point:
\be
\epsilon_{abs \; i,j}^{n} = |\tilde{\phi}^{n} - \tilde{\phi}^{n-1}|
\ee
This value is used in the program to limit the number of iterations the numerical process undertakes and also is used to 'lock' converged points to reduce calculations per iteration.
\subsection{Convergence Limit}
Within the program, the absolute convergence of the numerical solution is used as an indicator of precision. The absolute convergence is found for every point on the potential grid to determine the convergence of the system.

The number of iterations the program carries out for a system can be made to depend on the convergence; the system can be iterated over until all the points meet a desired absolute convergence $\epsilon$. This desired absolute convergence is taken from the command line arguments and is a value typically between $10^{-3}$ and $10^{-12}$. When the absolute convergence at every point has fallen below the desired absolute convergence, $\epsilon_{abs} < \epsilon$, the system is deemed to have converged. 

The successive over-relaxation method iterates within a while loop. This while loop is conditioned so that it loops through the potential grid with the successive over-relaxation method until the number of converged points is equal to the number of points on the grid (number of pixels in the image). The convergence count is set to zero at the start of each iteration, and it is incremented by one for every point that has a absolute convergence less than the desired absolute convergence.
