BMP (bitmap) files are uncompressed image files with three bytes of memory allocated to each pixel. The three bytes of memory allow each pixel to store a value for the colour scale for three different colours: blue, green and red. The colour scale runs from 0 to 255 - 0 being no contribution from that colour to the pixels colour and 255 being maximum contribution - and allows the pixel to have a range of 16,777,216 individual variations in colour. Although BMP files take up large quantities of memory per pixel as they are uncompressed, this allows them to be manipulated easily using other programs as the files do not need to be uncompressed before the information they hold can be utilised and no information is lost due to compressing and uncompressing. This gives them an advantage over other common image file types such as PNG, GIF and JPEG files which are all compressed. Due to the simplicity of the systems the images do not need to be large to store the boundary conditions so the quantity of memory is not a problem. 
\\
The potential boundaries for each system can be drawn onto a blank BMP file where the white pixels are the areas of unsolved potential and the different coloured areas are areas of different potentials where each colour has an assigned value of potential. For the systems given, there are only a maximum of four different values of fixed potentials and therefore only four different colours are necessary to represent them. The colours green, red, blue and black were chosen to represent these potentials as the strengths of green, blue and red can be determined comparing which colour scale is largest, and black is simply the absence of any values for the colour scales. The graphical representations of each of the systems are found in the appendix.
\\
File 'bitmap\textunderscore cpp' is used to open the bitmap files, before analysing individual pixels colours and mapping their location onto arrays of the same dimensions as the pixel array: one array stores the values of the potential of the boundaries and the unsolved areas (initially set to zero) and the other stores boolean elements set to true at boundary locations and false at unsolved areas to distinguish the two. To extract the information from the BMP files, a library of commands called 'bitmap\textunderscore image.hpp' (Author:  Arash Partow - 2002) was used. This allowed the program to open the BMP file using command:
\begin{align}
bitmap\textunderscore image \; image('file \;name')
\end{align}
Where the file name is taken from a command line argument. Integers were given values equal to the width and the height of the image using commands:
\begin{align}
int\; height = image.height(),\; width = image.width()
\end{align}
As the pixel array is looped through and the colour scales for each pixel is found using:
\begin{align}
image.get\textunderscore pixel(p,q,red,green,blue)
\end{align}
where $p$ and $q$ are the coordinates of the pixel and red, green and blue are the variables assigned with the value of the corresponding colour scale.
