BMP (bitmap) files are uncompressed image files. By being uncompressed, BMP files take up large quantities of memory but also allows them to be easily manipulated  using other programs. This gives them an advantage over other common image file types such as PNG, GIF and JPEG files which all require being uncompressed before their information can be extracted. In a BMP file, there are three bytes of memory for each pixel to store a value for the colour scale for three different colours: blue, green and red. The colour scale runs from 0 to 255 (0 being no contribution from that colour to the pixels colour and 255 being maximum contribution) and allows the pixel to have a range of 16,777,216 individual variations in colour.
\\
The potential boundaries for each system can be drawn onto a blank BMP file where the white pixels are the areas of unsolved potential and the different coloured areas are areas of different potentials where each colour has an assigned value of potential. For the systems given, there are only a maximum of four different values of fixed potentials and therefore only four different colours are necessary to represent them. The colours green, red, blue and black were chosen to represent these potentials as the strengths of green, blue and red can be determined comparing which colour scale is largest, and black is simply the absence of any values for the colour scales. The graphical representations of each of the systems are found in the appendix.
\\
File 'bitmap\textunderscore cpp' is used to extract the boundaries from the bitmap files. The file analyses each individual pixel colours and maps their location onto arrays of the same dimensions as the pixel array: the potential grid stores the potential of the boundaries and the unsolved areas (initially set to zero) and the bool mask which stores boolean elements set to true at boundary locations and false at unsolved areas to keep boundaries constant. To extract the information from the BMP files, a library of commands called 'bitmap\textunderscore image.hpp' (Author:  Arash Partow - 2002) was used. This allowed the program to open the BMP file using command:
\begin{listing}
bitmap_image image('file name')
\end{listing}
Where the file name is taken from a command line argument. Integers were given values equal to the width and the height of the image using commands:
\begin{listing}}
int height = image.height(), width = image.width()|
\end{listing}
As the pixel array is looped through and the colour scales for each pixel is found using:
\begin{listing}
image.get_pixel(p,q,red,green,blue)
\end{listing}
where $p$ and $q$ are the coordinates of the pixel and red, green and blue are the variables assigned with the value of the corresponding colour scale. The colour of the pixel is determined by comparing the colour scales. If the red colour scale is greater than the blue and green colour scales, then the point on the potential grid is given the potential from the command line arguments assigned to red pixels. The same applies to green and blue. If the colour scales all equal zero, then the colour of the pixel is said to be black which has an assigned value of potential also. As every system can be solved with a minimum of four different areas of constant potential, only colours green, red, blue and black can be assigned values of potential.
