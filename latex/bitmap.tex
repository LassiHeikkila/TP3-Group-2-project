Unlike PNG, GIF and JPEG image files, BMP files are uncompressed allowing them to be manipulated easily by programs. In a BMP file, there are three bytes of memory for each pixel to store a value for the colour scale for three different colours: blue, green and red. The colour scale runs from 0 to 255 (0 being no contribution from that colour to the pixels colour and 255 being maximum contribution) and allows the pixel to have a range of 16,777,216 individual variations in colour.

The potential boundaries for each system were drawn onto a blank BMP file where coloured areas are areas of constant potential and blank white pixels are areas of unsolved potential. Each colour has a different value of potential to assign to the points it covers. For the systems examined, only a minimum of four different potentials were needed and therefore were represented by the colours black, red, green and blue. The potential assigned to each colour was taken off the command line arguments. The strengths of green, blue and red can be determined comparing which colour scale is largest, and black is simply the absence of any values for the colour scales (BMP files of both the systems are found in the Appendix~\ref{app:bitmaps}).

\lstinline|eStatics| analyses each pixels colour and maps boundary conditions onto arrays of the same dimensions as the pixel array, these arrays are: the potential grid of \lstinline|double| values stores the potential of the boundaries and the unsolved areas (initially set to zero) and the \lstinline|bool| mask which stores boolean elements set to true at boundary locations and false at unsolved areas to keep boundaries constant. To extract the information from the BMP files, a library of commands called 'bitmap\textunderscore image.hpp'~\cite{arash} was used. This allowed the program to open the BMP file  by taking the file name from a command line argument before extracting the dimensions of the image and assigning the width and height to \lstinline|integer| values.

The pixel array is looped through and \lstinline|integer| are assigned with the value of the colour scale. The colour of the pixel is determined by comparing the colour scales; if the red colour scale is greater than the blue and green colour scales, then the point on the potential grid is given the potential from the command line arguments assigned to red pixels and \emph{vice-versa} for green and blue. If the colour scales all equal zero, then the colour of the pixel is said to be black. The value on the potential grid with the same location as the pixel is assigned with the value of potential associated with the pixels colour. Points on the boolean array at the locations of coloured pixels is set to true.
