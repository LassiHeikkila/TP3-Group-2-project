		A problem with iterative algorithms such as the Finite Difference Method (FDM) is that the initial boundary conditions can be eroded by successive averaging. To counteract this, these boundaries must either be rewritten with each iteration --- with significant costs in terms of time due to extra read/write cycles being required, and RAM usage due to a copy of the initial grid having to be kept in memory --- or write-protected for the duration of the program. For the latter, which can avoid or reduce the noted costs of the former, the boolean data type is ideally suited.
		
		\subsubsection{Boolean Mask}
		
		Boolean is a data type with only two possible values; true or false. As such, each boolean --- or \lstinline|bool| --- requires very little memory. Typically, a boolean variable is the smallest possible variable in terms of RAM requirements. This makes them ideal for use in the write-protection of boundary elements, as it drastically lowers the overhead in this area. In order to utilise this, a boolean array of equal dimension to the system array is created whilst the user-provided bitmap image is being analysed. If an element, or pixel, of the system is defined as a boundary element the corresponding element in the boolean mask is set to true. All non-boundary elements are set to false.
		
		Next, whilst the solver is running through the system array point-by-point, it first checks the corresponding element in the mask. If this boolean is found to be true, the solver ignores that element. This saves time, as no further calculation needs to take place on that element.
